\chapter{Introduction}
\label{ch:into} % This how you label a chapter and the key (e.g., ch:into) will be used to refer this chapter ``Introduction'' later in the report. 
% the key ``ch:into'' can be used with command \ref{ch:intor} to refere this Chapter.

\textbf Stockbrokers, traders, and investors that purchase, sell, or share transactions collectively make up the stock market. Because so many businesses post their stock lists online, investors find their stocks appealing \cite{bhattacharjee2019stock}. The size of the markets and the speed at which deals are completed now make it impractical for investors to use their own experience, which they once used to discern market trends. A nation's economic status affects a number of industries, including investment banking, metals, agriculture, and finance, either directly or indirectly. The fundamental law of supply and demand dictates that the growth of these industries depends on their volatility. Due to the fact that investors have been attempting various methods since the 16s in order to increase the returns on their investments by learning about various firms \cite{mehta2021stock}. There is a noticeable increase in demand for stock market. Because of the exceptional revenues, as we all know, it has been in the news for a long time. Accurately predicting future financial outcomes is the main goal of stock price prediction. Machine Learning algorithms have shown promise in a number of sectors in recent years, and as a result, many traders are using these strategies in their particular domains. 


An accounting forecast can be viewed as a difficult time-varying prediction. A transformer in deep learning is a model that uses the self-attention process and weights each incoming data component differently based on its significance. Transformers are intended to evaluate sequential input data, including plain language, much like recurrent neural networks (RNNs). But transformers handle the whole input all at once, unlike RNNs. The statistical prediction instruments that have been most widely used are ARIMA models. Though regrettably, not all future occurrences are predicted by this model with the same level of accuracy. This is because, following three or four predicted values, the forecasts converge to the series mean. Usually, the maximum likelihood model or the least squares estimation model are used to estimate the parameters of ARIMA models. In several disciplines, most notably economics, artificial neural networks (ANNs) have been utilized for prediction. 

After being first presented by a group of researchers at Google Brain in 2017, transformers are quickly displacing RNN models, such as extended short-term memory, as the preferred model for problems with natural language processing (LSTM). White and Safi (2016) demonstrated that when the ANN model's linking function was nonlinear, it outperformed the ARIMA model. But the ARIMA model did better than the ANN model when its linking function was linear. This is not unexpected given that the ARIMA model was created especially for this circumstance \cite{safi2021hybrid}. The hybrid model that is employed is a blend of the autoregressive integrated moving average (ARIMA) model, the nonlinear artificial neural network (ANN), and the Error, Trend, and Seasonality (ETS) models. It is a linear stochastic model. The hybrid model outperforms the linear and nonlinear models, according to data from earlier investigations. For instance, the hybrid model was shown to have the highest accuracy when used to forecast drought in India using a standardized precipitation index series, in comparison to separate stochastic and ANN models.

We use the Y-finance dataset to forecast a single ticker, taking into account several factors such as volume, high, close, and open. Utilizing nine tickers, we are able to achieve prediction characteristics whereby the tenth ticker is anticipated by utilizing all nine ticker values. To achieve these kinds of significant modifications, we approximate a method wherein the model's prediction weights produce the particular shift in testing and prediction values. Ultimately, based on the kind of algorithm selected and developed, a general comparison analysis using various time instances is confirmed. 

%%%%%%%%%%%%%%%%%%%%%%%%%%%%%%%%%%%%%%%%%%%%%%%%%%%%%%%%%%%%%%%%%%%%%%%%%%%%%%%%%%%
\section{Background}
\label{sec:into_back}
Describe to a reader the context of your project. That is, what is your project and what its motivation. Briefly explain the major theories, applications, and/or products/systems/algorithms whichever is relevant to your project.

\textbf{Cautions:} Do not say you choose this project because of your interest, or your supervisor proposed/suggested this project, or you were assigned this project as your final year project. This all may be true, but it is not meant to be written here.

%%%%%%%%%%%%%%%%%%%%%%%%%%%%%%%%%%%%%%%%%%%%%%%%%%%%%%%%%%%%%%%%%%%%%%%%%%%%%%%%%%%
\section{Problem statement}
\label{sec:intro_prob_art}
This section describes the investigated problem in detail. You can also have a separate chapter on ``Problem articulation.''  For some projects, you may have a section like ``Research question(s)'' or ``Research Hypothesis'' instead of a section on ``Problem statement.'

%%%%%%%%%%%%%%%%%%%%%%%%%%%%%%%%%%%%%%%%%%%%%%%%%%%%%%%%%%%%%%%%%%%%%%%%%%%%%%%%%%%
\section{Aims and objectives}
\label{sec:intro_aims_obj}
Describe the ``aims and objectives'' of your project. 

\textbf{Aims:} The aims tell a read what you want/hope to achieve at the end of the project. The  aims define your intent/purpose in general terms.  

\textbf{Objectives:} The objectives are a set of tasks you would perform in order to achieve the defined aims. The objective statements have to be specific and measurable through the results and outcome of the project.



%%%%%%%%%%%%%%%%%%%%%%%%%%%%%%%%%%%%%%%%%%%%%%%%%%%%%%%%%%%%%%%%%%%%%%%%%%%%%%%%%%%
\section{Solution approach}
\label{sec:intro_sol} % label of Org section
Briefly describe the solution approach and the methodology applied in solving the set aims and objectives.

Depending on the project, you may like to alter the ``heading'' of this section. Check with you supervisor. Also, check what subsection or any other section that can be added in or removed from this template.

\subsection{A subsection 1}
\label{sec:intro_some_sub1}
You may or may not need subsections here. Depending on your project's needs, add two or more subsection(s). A section takes at least two subsections. 

\subsection{A subsection 2}
\label{sec:intro_some_sub2}
Depending on your project's needs, add more section(s) and subsection(s).

\subsubsection{A subsection 1 of a subsection}
\label{sec:intro_some_subsub1}
The command \textbackslash subsubsection\{\} creates a paragraph heading in \LaTeX.

\subsubsection{A subsection 2 of a subsection}
\label{sec:intro_some_subsub2}
Write your text here...

%%%%%%%%%%%%%%%%%%%%%%%%%%%%%%%%%%%%%%%%%%%%%%%%%%%%%%%%%%%%%%%%%%%%%%%%%%%%%%%%%%%
\section{Summary of contributions and achievements} %  use this section 
\label{sec:intro_sum_results} % label of summary of results
Describe clearly what you have done/created/achieved and what the major results and their implications are. 


%%%%%%%%%%%%%%%%%%%%%%%%%%%%%%%%%%%%%%%%%%%%%%%%%%%%%%%%%%%%%%%%%%%%%%%%%%%%%%%%%%%
\section{Organization of the report} %  use this section
\label{sec:intro_org} % label of Org section
Describe the outline of the rest of the report here. Let the reader know what to expect ahead in the report. Describe how you have organized your report. 

\textbf{Example: how to refer a chapter, section, subsection}. This report is organised into seven chapters. Chapter~\ref{ch:lit_rev} details the literature review of this project. In Section~\ref{ch:method}...  % and so on.

\textbf{Note:}  Take care of the word like ``Chapter,'' ``Section,'' ``Figure'' etc. before the \LaTeX command \textbackslash ref\{\}. Otherwise, a  sentence will be confusing. For example, In \ref{ch:lit_rev} literature review is described. In this sentence, the word ``Chapter'' is missing. Therefore, a reader would not know whether 2 is for a Chapter or a Section or a Figure.

