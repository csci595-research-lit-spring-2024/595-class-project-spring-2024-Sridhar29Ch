\chapter{Literature Review}
\label{ch:lit_rev} %Label of the chapter lit rev. The key ``ch:lit_rev'' can be used with command \ref{ch:lit_rev} to refer this Chapter.

Bitcoin, a decentralized digital currency, has become more well-known as a substitute for stock market investing. Numerous dangers and factors have the potential to impact the stock market. However, bitcoin is one cryptocurrency that has been steadily increasing in value over the past several years, despite the fact that the value of other cryptocurrencies might suddenly crash without anybody being able to explain why. The stock market requires forecasts about bitcoin from both a human analyst and an automated software because to its volatility. 

Traders are eager to capitalize on the opportunities presented by the thriving Indian stock market at the moment. Because of this, it is important to estimate the Nifty 50's value accurately while trading stocks. The futures and options (F\&O) market component is one of the fundamental indicators of the Nifty 50.Machine learning techniques are used for more precise prediction. The forward neural network, the generalized regression neural network, the radial basis neural network, and the precise radial basis neural network are the four machine-learning-based neural network models we discuss in this study. By comparing the Nifty 50 value with a minimum error rate, four models were offered in the study effort to anticipate the degree of accuracy. In order to finish the research, samples were gathered from the Yahoo Finance website spanning the previous nine years, from January 2013 to June 2021. The original beginning value, day high, day low, and total volumes traded on that day are all considered input characteristics for the Nifty 50 value of the stock market. An output parameter is utilized, which is the day's closing value. The testing results demonstrate that the radial basis neural network model outperforms the other three models with an accuracy of 98.17\%, making it the best model for predicting the Nifty 50 stock market perspectives \citep{lamba2021comparative}.

After being first presented by a group of researchers at Google Brain in 2017, transformers are 
quickly displacing RNN models, such as extended short-term memory, as the preferred model for 
problems with natural language processing (LSTM). White and Safi (2016) demonstrated that when 
the ANN model's linking function was nonlinear, it outperformed the ARIMA model. But the ARIMA 
model did better than the ANN model when its linking function was linear. This is not 
unexpected given that the ARIMA model was created especially for this circumstance 
\citep{safi2021hybrid}. The hybrid model that is employed is a blend of the autoregressive 
integrated moving average (ARIMA) model, the nonlinear artificial neural network (ANN), and the 
Error, Trend, and Seasonality (ETS) models. It is a linear stochastic model. The hybrid model 
outperforms the linear and nonlinear models, according to data from earlier investigations. For 
instance, the hybrid model was shown to have the highest accuracy when used to forecast drought 
in India using a standardized precipitation index series, in comparison to separate stochastic 
and ANN models.

Big data analytics are used to accurately forecast and analyze large data collections. They enable the discovery of valuable information from massive data sets that might otherwise be lost. This research proposes a durable Cloudera-Hadoop-based data pipeline strategy for executing studies for any scale and kind of data, in which chosen US equities are analyzed to estimate daily gains using real-time data from Yahoo Finance. Stocks from the US stock market are chosen, and data on their daily gains is divided into training and test data sets so that Spark's Machine Learning module can forecast the stocks with the highest daily gains. The Apache Hadoop big-data platform is available for handling enormous data collections via distributed storage and processing \citep{peng2019stocks}.
Because of the market's high level of intricacy and volatility, forecasting its behavior is challenging. Researchers are investigating a number of stock market forecasting methodologies to improve accuracy. In this article, we propose using an SBLSTM (Stacked Bidirectional Long Short-Term Memory) network to anticipate stock market movements. To create a deep neural network model, the suggested SBLSTM stacks three bidirectional LSTM networks on top of each other. This method has the potential to increase prediction performance when it comes to stock price forecasts. Unlike LSTM-based approaches, the SBLSTM offered collects temporal data in both forward and backward ways using bidirectional LSTM layers. This method controls the long-term interdependence of the stock market's expected and historical values. The suggested SBLSTM's performance is evaluated using six different Yahoo Finance datasets. Furthermore, the proposed SBLSTM is compared to existing cutting-edge approaches based on root mean square error. According to empirical testing on the dataset \citep{lim2021stacked}, the proposed SBLSTM outperforms state-of-the-art approaches.

Volatility clustering is an important factor that influences how stock markets work. However, developing credible models for accurately anticipating future stock price volatility is a difficult research challenge to address. We provide multiple volatility models based on the generalized autoregressive conditional heteroscedasticity (GARCH) framework for describing the volatility of ten equities listed on India's National Stock Exchange (NSE). The stocks were offered by the Indian economy's banking and car industries, and they have a substantial influence on the sectoral indexes maintained by their respective industries on NSE. Historical stock price data from January 1, 2010, to April 30, 2021 were extracted from the Yahoo Finance website using the Python computer language's Data Reader API. This operation began on January 1, 2010. The training data is used to create and fine-tune the GARCH modules, while the out-of-sample data is used to test the models and evaluate their performance. According to the results, asymmetric GARCH models forecast future stock price volatility more precisely \citep{sen2021volatility}.



% PLEAE CHANGE THE TITLE of this section
\section{Example of in-text citation of references in \LaTeX} 
% Note the use of \cite{} and \citep{}
The references in a report relate your content with the relevant sources, papers, and the works of others. To include references in a report, we \textit{cite} them in the texts. In MS-Word, EndNote, or MS-Word references, or plain text as a list can be used. Similarly, in \LaTeX, you can use the ``thebibliography'' environment, which is similar to the plain text as a list arrangement like the MS word. However, In \LaTeX, the most convenient way is to use the BibTex, which takes the references in a particular format [see references.bib file of this template] and lists them in style [APA, Harvard, etc.] as we want with the help of proper packages.    

These are the examples of how to \textit{cite} external sources, seminal works, and research papers. In \LaTeX, if you use ``\textbf{BibTex}'' you do not have to worry much since the proper use of a bibliographystyle package like ``agsm for the Harvard style'' and little rectification of the content in a BiBText source file [In this template, BibTex are stored in the ``references.bib'' file], we can conveniently generate  a reference style. 

Take a note of the commands \textbackslash cite\{\} and \textbackslash citep\{\}. The command \textbackslash cite\{\} will write like ``Author et al. (2019)'' style for Harvard, APA and Chicago style. The command \textbackslash citep\{\} will write like ``(Author et al., 2019).'' Depending on how you construct a sentence, you need to use them smartly. Check the examples of \textbf{in-text citation} of sources listed here [This template recommends the \textbf{Harvard style} of referencing.]:
\begin{itemize}
    \item \cite{lamport1994latex} has written a comprehensive guide on writing in \LaTeX ~[Example of \textbackslash cite\{\} ].
    \item If \LaTeX~is used efficiently and effectively, it helps in writing a very high-quality project report~\citep{lamport1994latex} ~[Example of \textbackslash citep\{\} ].   
    \item A detailed APA, Harvard, and Chicago referencing style guide are available in~\citep{uor_refernce_style}.
\end{itemize}

\noindent 
Example of a numbered list:
\begin{enumerate}
    \item \cite{lamport1994latex} has written a comprehensive guide on writing in \LaTeX.
    \item If \LaTeX is used efficiently and effectively, it helps in writing a very high-quality project report~\citep{lamport1994latex}.   
\end{enumerate}

% PLEAE CHANGE THE TITLE of this section
\section{Example of ``risk'' of unintentional plagiarism}
Using other sources, ideas, and material always bring with it a risk of unintentional plagiarism. 

\noindent
\textbf{\color{red}MUST}: do read the university guidelines on the definition of plagiarism as well as the guidelines on how to avoid plagiarism~\citep{uor_plagiarism}.




% A possible section of you chapter
\section{Critique of the review} % Use this section title or choose a betterone
Describe your main findings and evaluation of the literature. ~\\

% Pleae use this section
\section{Summary} 
Write a summary of this chapter~\\
