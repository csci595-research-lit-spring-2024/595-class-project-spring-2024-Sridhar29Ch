\chapter{Results}
\label{ch:results}
The results chapter tells a reader about your findings based on the methodology you have used to solve the investigated problem. For example: 
\begin{itemize}
    \item If your project aims to develop a software/web application, the results may be the developed software/system/performance of the system, etc., obtained using a relevant methodological approach in software engineering. 
    
    \item If your project aims to implement an algorithm for its analysis, the results may be the performance of the algorithm obtained using a relevant experiment design. 
    
    \item If your project aims to solve some problems/research questions over a collected dataset, the results may be the findings obtained using the applied tools/algorithms/etc. 
\end{itemize}
Arrange your results and findings in a logical sequence. 



\section{A section}

...

\clearpage
\section{Example of a Table in \LaTeX}
Table~\ref{tab:_ex_tab} is an example of a table created using the package \LaTeX  ``booktabs.'' do check the link: \href{https://en.wikibooks.org/wiki/LaTeX/Tables}{wikibooks.org/wiki/LaTeX/Tables} for more details. A table should be clean and readable. Unnecessary horizontal lines and vertical lines in tables make them unreadable and messy. The example in Table~\ref{tab:_ex_tab} uses a minimum number of liens (only necessary ones). Make sure that the top rule and bottom rule (top and bottom horizontal lines) of a table are present. 

\begin{table}[h!]
    \centering
    \caption{Example of a table in \LaTeX}
    \label{tab:_ex_tab}
    \begin{tabular}{llr}     
        \toprule
        \multicolumn{2}{c}{Bike} \\
        \cmidrule(r){1-2}
        Type    &  Color & Price (\pounds) \\
        \midrule
        Electric    & black   & 700   \\
        Hybrid      & blue    & 500   \\
        Road        & blue    & 300   \\
        Mountain    & red     & 300   \\
        Folding     & black   & 500   \\
        \bottomrule
    \end{tabular}
\end{table}

\section{Example of captions style}

\begin{itemize}
    \item The \textbf{caption of a Figure (artwork) goes below} the artwork (Figure/Graphics/illustration). See example artwork in Figure~\ref{fig:chart_a}. 
    \item  The \textbf{caption of a Table goes above} the table. See the example in Table~\ref{tab:_ex_tab}.
    \item  The \textbf{caption of an Algorithm goes above} the algorithm. See the example in Algorithm~\ref{algo:algo_example}.
    \item The \textbf{caption of a Listing goes below} the Listing  (Code snippet). See example listing in Listing~\ref{list:python_code_ex}. 
\end{itemize} 





\section{Summary}
Write a summary of this chapter.



